% !TeX root = main.tex

\subsection{3R Inverse Kinematics Animation}

\subsubsection{Forward Kinematics}
\label{app:3r-fk-code}

The code to generate the position of each joint as well as end effector is given below. The function \texttt{jfk\_min\_3r} calculates and returns the $(x, y, \theta)$ values of joint 2, joint 3 and the end effector.

\lstinputlisting[language=python]{../python/fk_3r.py}

\subsubsection{Inverse Kinematics}
\label{app:3r-ik-code}

The code to generate the joint angles is given below. The function \texttt{ik\_3r} calculates and returns the $(\theta_1, \theta_2, \theta_3)$ that are the joint values for the manipulator.

\lstinputlisting[language=python]{../python/ik_3r.py}

Equations \ref{eq:3r-ik-t1}, \ref{eq:3r-ik-t2} and \ref{eq:3r-ik-t3} are used for the implementation above.

\subsubsection{3R IK Animation}
\label{app:3r-ik-anim}

The animation for IK (of a 3R manipulator) is generated by the code below. This depends upon code in Appendix \ref{app:3r-fk-code} and \ref{app:3r-ik-code}. Set the variables \texttt{ik\_start}, \texttt{ik\_end} and \texttt{ik\_th} accordingly.

\lstinputlisting[language=python]{../python/ik_3r_anim.py}

\subsection{3R Dynamic Modeling}
\label{app:3r-dyn-model-pycode}

The code to generate and dynamic model (equations) of the 3R manipulator is presented in this section.

\lstinputlisting[language=python, lastline=200]{../python/dyn_3r.py}

The output of this program is

\begin{displayquote}
    The mass matrix is symmetric (M = M.T) \\
    M\_dot - 2C is not skew symmetric \\
    M\_dot - 2C\_cris is skew symmetric
\end{displayquote}

The program also displays the value of $\textup{ssm.T + ssm}$, where $\textup{ssm} = \mathbf{\dot{M} - 2C}$. Run this in \href{https://code.visualstudio.com/docs/python/jupyter-support-py}{VSCode Python Interactive} for best results.

The matrix $\mathbf{\dot{M} - 2C}$ turns out to be

\begin{equation}
    \textup{ssm} = \mathbf{\dot{M} - 2C} = \begin{bmatrix}
        S_{11} & S_{12} & S_{13} \\
        S_{21} & S_{22} & S_{23} \\
        S_{31} & S_{32} & S_{33}
    \end{bmatrix}
    \label{eq:app-3r-dyn-ssm-mat}
\end{equation}

Where

\begin{align}
    \begin{split}
        S_{11} =& 2 \dot{q}_2 l_{1} l_{2} m_{3} s_{2} + 2 \dot{q}_2 l_{1} m_{2} r_{2} s_{2} + 2 \dot{q}_3 l_{2} m_{3} r_{3} s_{3} + 2 l_{1} m_{3} r_{3} s_{23} \left(\dot{q}_2 + \dot{q}_3\right)
    \end{split}
    \nonumber \\
    \begin{split}
        S_{12} =& \dot{q}_2 l_{1} l_{2} m_{3} s_{2} + \dot{q}_2 l_{1} m_{2} r_{2} s_{2} + 2 \dot{q}_3 l_{2} m_{3} r_{3} s_{3} + l_{1} m_{3} r_{3} s_{23} \left(\dot{q}_2 + \dot{q}_3\right)
    \end{split}
    \nonumber \\
    \begin{split}
        S_{13} =& m_{3} r_{3} \left(\dot{q}_3 l_{2} s_{3} + l_{1} s_{23} \left(\dot{q}_2 + \dot{q}_3\right)\right)
    \end{split}
    \\
    \begin{split}
        S_{21} =& - 2.0 \dot{q}_1 l_{1} l_{2} m_{3} s_{2} - 2.0 \dot{q}_1 l_{1} m_{2} r_{2} s_{2} - 2.0 \dot{q}_1 l_{1} m_{3} r_{3} s_{23} + 2.0 \dot{q}_3 l_{2} m_{3} r_{3} s_{3}
    \end{split}
    \nonumber \\
    \begin{split}
        S_{22} =& - \dot{q}_1 l_{1} \left(l_{2} m_{3} s_{2} + m_{2} r_{2} s_{2} + m_{3} r_{3} s_{23}\right) + 2 \dot{q}_3 l_{2} m_{3} r_{3} s_{3}
    \end{split}
    \nonumber \\
    \begin{split}
        S_{23} =& m_{3} r_{3} \left(- \dot{q}_1 l_{1} s_{23} + \dot{q}_3 l_{2} s_{3}\right)
    \end{split}
    \\
    \begin{split}
        S_{31} =& - 2.0 m_{3} r_{3} \left(\dot{q}_1 l_{1} s_{23} + \dot{q}_1 l_{2} s_{3} + \dot{q}_2 l_{2} s_{3}\right)
    \end{split}
    \nonumber \\
    \begin{split}
        S_{32} =& - m_{3} r_{3} \left(1.0 \dot{q}_1 \left(l_{1} s_{23} + 2 l_{2} s_{3}\right) + 2.0 \dot{q}_2 l_{2} s_{3}\right)
    \end{split}
    \nonumber \\
    \begin{split}
        S_{33} =& - 1.0 m_{3} r_{3} \left(\dot{q}_1 \left(l_{1} s_{23} + l_{2} s_{3}\right) + \dot{q}_2 l_{2} s_{3}\right)
    \end{split}
\end{align}

That is not a skew-symmetric matrix, however, using the coriolis matrix obtained using Cristoffel Symbols, we get the matrix $\mathbf{\dot{M}} - 2\mathbf{C}_{\textup{cris}}$ as $\textup{ssm}_{\textup{cris}}$

\begin{equation}
    \textup{ssm}_{\textup{cris}} = \mathbf{\dot{M}} - 2\mathbf{C}_{\textup{cris}} = \begin{bmatrix}
        B_{11} & B_{12} & B_{13} \\
        B_{21} & B_{22} & B_{23} \\
        B_{31} & B_{32} & B_{33}
    \end{bmatrix}
    \label{eq:app-3r-dyn-ssm-cris-mat}
\end{equation}

Where

\begin{align}
    \begin{split}
        B_{11} =& 0
    \end{split}
    \nonumber \\
    \begin{split}
        B_{12} =& l_{1} \left(2.0 \dot{q}_1 l_{2} m_{3} s_{2} + 2.0 \dot{q}_1 m_{2} r_{2} s_{2} + 2.0 \dot{q}_1 m_{3} r_{3} s_{23} + 1.0 \dot{q}_2 l_{2} m_{3} s_{2} + 1.0 \dot{q}_2 m_{2} r_{2} s_{2} + \right. \\
            & \left. 1.0 \dot{q}_2 m_{3} r_{3} s_{23} + 1.0 \dot{q}_3 m_{3} r_{3} s_{23}\right)
    \end{split}
    \nonumber \\
    \begin{split}
        B_{13} =& m_{3} r_{3} \left(- \dot{q}_3 l_{2} s_{3} - l_{1} s_{23} \left(\dot{q}_2 + \dot{q}_3\right) + 2.0 \left(l_{1} s_{23} + l_{2} s_{3}\right) \left(\dot{q}_1 + \dot{q}_2 + \dot{q}_3\right)\right)
    \end{split}
    \\
    \begin{split}
        B_{21} =& - l_{1} \left(2 \dot{q}_1 \left(l_{2} m_{3} s_{2} + m_{2} r_{2} s_{2} + m_{3} r_{3} s_{23}\right) + \dot{q}_2 l_{2} m_{3} s_{2} + \dot{q}_2 m_{2} r_{2} s_{2} + m_{3} r_{3} s_{23} \left(\dot{q}_2 + \dot{q}_3\right)\right)
    \end{split}
    \nonumber \\
    \begin{split}
        B_{22} =& 0
    \end{split}
    \nonumber \\
    \begin{split}
        B_{23} =& l_{2} m_{3} r_{3} s_{3} \left(2.0 \dot{q}_1 + 2.0 \dot{q}_2 + 1.0 \dot{q}_3\right)
    \end{split}
    \\
    \begin{split}
        B_{31} =& - m_{3} r_{3} \left(2 \dot{q}_1 \left(l_{1} s_{23} + l_{2} s_{3}\right) + 2.0 \dot{q}_2 l_{2} s_{3} + \dot{q}_3 l_{2} s_{3} + l_{1} s_{23} \left(\dot{q}_2 + \dot{q}_3\right)\right)
    \end{split}
    \nonumber \\
    \begin{split}
        B_{32} =& - l_{2} m_{3} r_{3} s_{3} \left(2.0 \dot{q}_1 + 2.0 \dot{q}_2 + \dot{q}_3\right)
    \end{split}
    \nonumber \\
    \begin{split}
        B_{33} =& 0
    \end{split}
\end{align}

This is a \textbf{skew-symmetric matrix} as its transpose is the negative of itself. This can be verified by the diagonal values being zero and the off diagonal terms being the negative of their transpose correspondence. Basically $\textup{ssm}_{\textup{cris}} + \textup{ssm}_{\textup{cris}}^\top = 0$.
