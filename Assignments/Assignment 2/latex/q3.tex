% !TeX root = main.tex

\section[A3: Axis-angle and SO(3)]{Axis angle convention}

\subsection*{Axis-angle to Rotation Matrix}

The rotation matrix, in terms of the Axis-angle convention is given by

\begin{equation}
    \mathbf{R} = \mathbf{I} + (\sin(\theta)) \left [ \mathbf{\hat{n}} \right ]_{\times} + (1-\cos(\theta)) \left [ \mathbf{\hat{n}} \right ]_{\times}^{2}
\end{equation}

Where $\theta$ is the angle of rotation about unit axis $\mathbf{\hat{n}}$ (that is, $n_x^2 + n_y^2 + n_z^2 = 1$). The $\left [ \mathbf{\hat{n}} \right ]_{\times}$ is the cross product matrix given by

\begin{align}
    \mathbf{\hat{n}} = \begin{bmatrix}
        n_x \\ n_y \\ n_z
        \end{bmatrix} &&
    \left [ \mathbf{\hat{n}} \right ]_{\times} = \begin{bmatrix}
        0 & -n_z & n_y \\
        n_z & 0 & -n_x \\
        -n_y & n_x & 0
        \end{bmatrix} &&
    \left [ \mathbf{\hat{n}} \right ]_{\times}^{2} = \begin{bmatrix}
        -n_z^2-n_y^2 & n_x n_y & n_x n_z \\
        n_y n_x & -n_x^2-n_z^2 & n_y n_z \\
        n_z n_x & n_z n_y & -n_y^2-n_x^2
        \end{bmatrix}
\end{align}

Substituting this to get $\mathbf{R}$ in the equations above, we get

\begin{align}
    \mathbf{R} &= \mathbf{I} + (\sin(\theta)) \left [ \mathbf{\hat{n}} \right ]_{\times} + (1-\cos(\theta)) \left [ \mathbf{\hat{n}} \right ]_{\times}^{2} 
    \label{eq:axang-to-rot-matrix}
    \\
    &= \begin{bmatrix}
        1 + (1-\cos(\theta)) (n_x^2 - 1) & -n_z \sin(\theta) + (1-\cos(\theta)) n_x n_y & n_y \sin(\theta) + (1-\cos(\theta)) n_x n_z \\
        n_z \sin(\theta) + (1-\cos(\theta)) n_y n_x & 1 + (1-\cos(\theta))(n_y^2 - 1) & -n_x \sin(\theta) + (1-\cos(\theta)) n_y n_z \\
        -n_y \sin(\theta) + (1-\cos(\theta)) n_z n_x & n_x \sin(\theta) + (1-\cos(\theta)) n_z n_y & 1 + (1-\cos(\theta)) (n_z^2 - 1)
        \end{bmatrix} \nonumber
\end{align}

\subsection[A3.1: SO(3) to Axis-angle]{Converting Rotation Matrix to Axis Angle}

To convert rotation matrix to axis-angle numbers (that is, to get $\mathbf{n}$ and $\theta$ from $\mathbf{R}$), we can refer to the equation \ref{eq:axang-to-rot-matrix} and backtrack. We get the following equations

\begin{align}
    \mathbf{R} = \begin{bmatrix}
        r_{11} & r_{12} & r_{13} \\
        r_{21} & r_{22} & r_{23} \\
        r_{31} & r_{32} & r_{33} \\
        \end{bmatrix} &&
    \theta = \arccos \left ( \frac{r_{11} + r_{22} + r_{33} - 1}{2} \right ) &&
    \mathbf{n} = \begin{bmatrix}
        n_x \\ n_y \\ n_z
        \end{bmatrix} = \frac{1}{2\sin(\theta)} \begin{bmatrix}
        r_{32} - r_{23} \\
        r_{13} - r_{31} \\
        r_{21} - r_{12}
        \end{bmatrix}
    \label{eq:rot-matrix-to-axang}
\end{align}

\subsubsection*{Validating $\theta$}

We can substitute for $\theta$ from Equation \ref{eq:rot-matrix-to-axang} and validate using Equation \ref{eq:axang-to-rot-matrix}.

\begin{equation}
    \begin{split}
        \theta &= \arccos \left ( \frac{r_{11} + r_{22} + r_{33} - 1}{2} \right ) \\
        \Rightarrow \theta &= \arccos \left ( \frac{1 + (1-\cos(\theta)) (n_x^2 - 1) + 1 + (1-\cos(\theta))(n_y^2 - 1) + 1 + (1-\cos(\theta)) (n_z^2 - 1) - 1}{2} \right ) \\
        \Rightarrow \theta &= \arccos \left ( \frac{2 + (1-\cos(\theta)) (n_x^2 - 1) + (1-\cos(\theta))(n_y^2 - 1) + (1-\cos(\theta)) (n_z^2 - 1)}{2} \right ) \\
        \Rightarrow \theta &= \arccos \left ( \frac{2 + (1-\cos(\theta))(n_x^2 + n_y^2 + n_z^2 - 3)}{2} \right ) \\
        \Rightarrow \theta &= \arccos \left ( \frac{2 + (1-\cos(\theta))(-2)}{2} \right ) = \arccos \left ( \frac{2 - 2 + 2 \cos(\theta)}{2} \right ) \\
        \Rightarrow \theta &= \arccos \left ( \cos(\theta) \right ) \Rightarrow \theta = \theta
    \end{split}
\end{equation}

This validates that the formula for $\theta$ in Equation \ref{eq:rot-matrix-to-axang} is correct.

\subsubsection*{Validating $\mathbf{n}$}

We can substitute for $\mathbf{n}$ from Equation \ref{eq:rot-matrix-to-axang} and validate using Equation \ref{eq:axang-to-rot-matrix}.

\begin{equation}
    \begin{split}
        \mathbf{n} &= \begin{bmatrix}
            n_x \\ n_y \\ n_z
            \end{bmatrix} = \frac{1}{2\sin(\theta)} \begin{bmatrix}
            r_{32} - r_{23} \\
            r_{13} - r_{31} \\
            r_{21} - r_{12}
            \end{bmatrix} \\
        \Rightarrow \mathbf{n} &= \frac{1}{2\sin(\theta)} \begin{bmatrix}
            \left [ n_x \sin(\theta) + (1-\cos(\theta)) n_z n_y \right ] - \left [ -n_x \sin(\theta) + (1-\cos(\theta)) n_y n_z \right ] \\
            \left [ n_y \sin(\theta) + (1-\cos(\theta)) n_x n_z \right ] - \left [ -n_y \sin(\theta) + (1-\cos(\theta)) n_z n_x \right ]\\
            \left [ n_z \sin(\theta) + (1-\cos(\theta)) n_y n_x \right ] - \left [ -n_z \sin(\theta) + (1-\cos(\theta)) n_x n_y \right ]
            \end{bmatrix} \\
        \Rightarrow \mathbf{n} &= \frac{1}{2\sin(\theta)} \begin{bmatrix}
            2 n_x \sin(\theta) \\
            2 n_y \sin(\theta) \\
            2 n_z \sin(\theta)
            \end{bmatrix} \\
        \Rightarrow \mathbf{n} &= \begin{bmatrix}
            n_x \\ n_y \\ n_z
            \end{bmatrix}
    \end{split}
\end{equation}

This validates that the formula for $\mathbf{n}$ in Equation \ref{eq:rot-matrix-to-axang} is correct.

\subsection[A3.2: Conversion functions]{Conversion functions}

The functions to convert a rotation matrix to axis-angle and axis-angle to rotation matrix are implemented in Appendix \ref{app:a3.2-axang-rotm-conv-code}.
