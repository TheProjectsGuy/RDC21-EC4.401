% !TeX root = main.tex

\section[A1: Euler ZYX]{Euler ZYX Convention}

Equations for rotation matrices from principal axis rotations
\begin{equation}
    \begin{split}
        \mathbf{R}(\hat{Z}, \theta) = \begin{bmatrix}
            \cos(\theta) & -\sin(\theta) & 0 \\
            \sin(\theta) & \cos(\theta) & 0 \\
            0 & 0 & 1
            \end{bmatrix} \\
        \mathbf{R}(\hat{Y}, \theta) = \begin{bmatrix}
            \cos(\theta) & 0 & \sin(\theta) \\
            0 & 1 & 0 \\
            -\sin(\theta) & 0 & \cos(\theta)
            \end{bmatrix} \\
        \mathbf{R}(\hat{X}, \theta) = \begin{bmatrix}
            1 & 0 & 0 \\
            0 & \cos(\theta) & -\sin(\theta) \\
            0 & \sin(\theta) & \cos(\theta)
            \end{bmatrix}
    \end{split}
\end{equation}

\subsection[A1.1: Euler to SO(3)]{Converting Euler angles to Rotation Matrix}

The Euler ZYX rotation is given by (consider $s_\theta = \sin(\theta)$ and $c_\theta = \cos(\theta)$)

\begin{equation}
    \begin{split}
        \mathbf{R}_{ZYX} \left ( \alpha, \beta, \gamma \right ) & = \mathbf{R}(\hat{Z}, \alpha) \mathbf{R}(\hat{Y}, \beta) \mathbf{R}(\hat{X}, \gamma) \\
        & = \begin{bmatrix}
            \cos(\alpha) & -\sin(\alpha) & 0 \\
            \sin(\alpha) & \cos(\alpha) & 0 \\
            0 & 0 & 1
            \end{bmatrix}
            \begin{bmatrix}
            \cos(\beta) & 0 & \sin(\beta) \\
            0 & 1 & 0 \\
            -\sin(\beta) & 0 & \cos(\beta)
            \end{bmatrix}
            \begin{bmatrix}
            1 & 0 & 0 \\
            0 & \cos(\gamma) & -\sin(\gamma) \\
            0 & \sin(\gamma) & \cos(\gamma)
            \end{bmatrix} \\
        & = \begin{bmatrix}
            c_\alpha c_\beta & -s_\alpha c_\gamma + s_\beta s_\gamma c_\alpha & s_\alpha s_\gamma + s_\beta c_\alpha c_\gamma \\
            s_\alpha c_\beta & s_\alpha s_\beta s_\gamma + c_\alpha c_\gamma & s_\alpha s_\beta c_\gamma - s_\gamma c_\alpha \\
            -s_\beta & s_\gamma c_\beta & c_\beta c_\gamma
            \end{bmatrix}
    \end{split}
    \label{eq:euzyx-to-rotm}
\end{equation}

A Python function that can do this is written in Appendix \ref{app:a1.1-eu2rm-code} and given with this document.

\subsection[A1.2: SO(3) to Euler]{Converting Rotation Matrix to Euler angles}

The Equation \ref{eq:euzyx-to-rotm} has to be reversed. The rotation matrix is given as

\begin{equation}
    \mathrm{R} = \begin{bmatrix}
        r_{11} & r_{12} & r_{13} \\
        r_{21} & r_{22} & r_{23} \\
        r_{31} & r_{32} & r_{33}
        \end{bmatrix}
    \nonumber
\end{equation}

Relating the terms from Equation \ref{eq:euzyx-to-rotm}, we have the following values for 

\begin{itemize}
    \item $\alpha$ being the angle of rotation about Z axis
    \item $\beta$ being the angle of rotation about Y axis
    \item $\gamma$ being the angle of rotation about X axis
\end{itemize}

In the equation below (in a generic setting)

\begin{equation}
    \begin{split}
        \alpha &= \textup{arctan2} \left ( r_{21}, r_{11} \right ) \\
        \beta &= \textup{arctan2} \left ( -r_{31}, \sqrt{r_{11}^2 + r_{21}^2} \right ) = \textup{arctan2} \left ( -r_{31}, \sqrt{r_{32}^2 + r_{33}^2} \right ) \\
        \gamma &= \textup{arctan2} \left ( r_{32}, r_{33} \right )
    \end{split}
    \label{eq:rotm-to-euzyx-norm}
\end{equation}

\subsubsection*{Singularity}

If the value of $\beta = \pm 90^{\circ}$, then $r_{21} = r_{11} = r_{32} = r_{33} = 0$. This makes resolving individual $\alpha$ and $\gamma$ impossible. The two cases are described below

\paragraph*{Case 1}
If the value of $\beta = 90^{\circ}$. The Equation \ref{eq:euzyx-to-rotm} basically becomes

\begin{equation}
    \mathbf{R}_{ZYX} \left ( \alpha, \beta = \frac{\pi}{2}, \gamma \right ) = \begin{bmatrix}
        0 & -\sin(\alpha-\gamma) & \cos(\alpha-\gamma) \\
        0 & \cos(\alpha-\gamma) & \sin(\alpha-\gamma) \\
        -1 & 0 & 0
        \end{bmatrix}
\end{equation}

\paragraph*{Case 2}
If the value of $\beta = -90^{\circ}$, the Equation \ref{eq:euzyx-to-rotm} becomes

\begin{equation}
    \mathbf{R}_{ZYX} \left ( \alpha, \beta = -\frac{\pi}{2}, \gamma \right ) = \begin{bmatrix}
        0 & -\sin(\alpha+\gamma) & -\cos(\alpha+\gamma) \\
        0 & \cos(\alpha+\gamma) & -\sin(\alpha+\gamma) \\
        1 & 0 & 0
        \end{bmatrix}
\end{equation}

\noindent
All the equations above are implemented as a Python function, code is presented in Appendix \ref{app:a1.2-rm2eu-code} and given with this document.
